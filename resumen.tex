\documentclass{article}

\usepackage{amssymb}
\usepackage[utf8]{inputenc}
\usepackage[T1]{fontenc}

\begin{document}
Dada una gráfica simple finita $G$, su gráfica de clanes $K(G)$ es la
gráfica de intersección de las subgráficas completas maximales de
$G$. Neumann-Lara demostró en 1976 que el operador de clanes preserva
el producto fuerte de gráficas, es decir
$K(G\boxtimes H)\cong K(G)\boxtimes K(G)$, y en 2000, Larrión y
Neumann-Lara mostraron que es posible definir un concepto de función
cubriente entre gráficas $f\colon G\to H$ análogo al estudiado en
topología algebraica, de tal modo que se obtenga una función cubriente
$K(f)\colon K(G)\to K(H)$.

En la presente plática mostraremos un concepto análogo al de haz
fibrado en topología algebraica, el cual generaliza al producto y las
funciones cubrientes de gráficas, y que es preservado bajo el operador
de clanes.
\end{document}
